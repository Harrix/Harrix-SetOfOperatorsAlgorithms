\subsection{Плоское скрещивание для вещественных векторов}\label{SetOfOperatorsAlgorithms:FlatCrossoverForReal}

Идентификатор: \textbf{FlatCrossoverForReal}.

Данный оператор скрещивания используется для вещественных векторов.

Пусть имеется два родителя (родительские хромосомы) $ \overline{Parent}^1 $ и $ \overline{Parent}^2$ длины $n$. Гены потомка генерируются как случайное вещественное число в границах соответствующих генов родителей:
\begin{align}
\label{SetOfOperatorsAlgorithms:eq:FlatCrossoverForReal}
&Crossover \left( \overline{Parent}^1, \overline{Parent}^2, DataOfCros\right)= \overline{Offspring}, \\
& \overline{Offspring}_i=random\left(\min\left(\overline{Parent}^1_i, \overline{Parent}^2_i \right),\max\left(\overline{Parent}^1_i, \overline{Parent}^2_i \right)  \right);\nonumber\\
&\overline{Offspring}\in X.\nonumber
\end{align}

$ DataOfCros $ не содержит каких-либо параметров относительно данного типа скрещивания.

В библиотеке \textbf{HarrixMathLibrary} данная селекция реализуется через функцию \textbf{HML\_FlatCrossoverForReal}:

\href{https://github.com/Harrix/HarrixMathLibrary}{https://github.com/Harrix/HarrixMathLibrary}

Данный оператор приводится в следующих источниках:

\begin{enumerate}
\item \cite{web:basegroup.ru:real_coded_ga} ---  \href{http://www.basegroup.ru/library/optimization/real_coded_ga/}{Непрерывные генетические алгоритмы - математический аппарат}.
\end{enumerate}