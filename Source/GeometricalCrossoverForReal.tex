\subsection{Геометрическое скрещивание для вещественных векторов}\label{SetOfOperatorsAlgorithms:GeometricalCrossoverForReal}

Идентификатор: \textbf{GeometricalCrossoverForReal}.

Данный оператор скрещивания используется для вещественных векторов.

Пусть имеется два родителя (родительские хромосомы) $ \overline{Parent}^1 $ и $ \overline{Parent}^2$ длины $n$. Гены первого потомка формируются как произведение генов первого родителя в степени $w$ и генов второго родителя в степени $\left( 1-w\right) $. Второй потом генерируется алогично, но степени меняются местами.  Из них выбирается случайно один потомок, который и передается в качестве результата оператора скрещивания. То есть скрещивание происходит по формулам:
\begin{align}
\label{SetOfOperatorsAlgorithms:eq:GeometricalCrossoverForReal}
&Crossover \left( \overline{Parent}^1, \overline{Parent}^2, DataOfCros\right)=Random \left(\left\lbrace \overline{Offspring}^1; \overline{Offspring}^2\right\rbrace  \right), \\
& \overline{Offspring}^1_i=\left( \overline{Parent}^1_i\right)^w \cdot\left( \overline{Parent}^2_i\right)^{1-w}  , i=\overline{1,n};\nonumber\\
& \overline{Offspring}^1_i=\left( \overline{Parent}^1_i\right)^{1-w} \cdot\left( \overline{Parent}^2_i\right)^w  , i=\overline{1,n};\nonumber\\
&\overline{Offspring}^1\in X, \overline{Offspring}^2\in X, w\in \left[ 0; 1\right] .\nonumber
\end{align}

Геометрическое скрещивание для вещественных векторов с возвращением добавляет в $ DataOfCros $ дополнительный параметр --- параметр скрещивания, который означает своеобразную долю какого-то родителя в потомке $ w $. Обычно выбирают значение этого параметра равное $ w=0.5 $.

\begin{equation}
DataOfCros=\left( \begin{array}{c} TypeOfCros \\ w \end{array} \right).
\end{equation}

В библиотеке \textbf{HarrixMathLibrary} данная селекция реализуется через функцию \textbf{HML\_GeometricalCrossoverForReal}:

\href{https://github.com/Harrix/HarrixMathLibrary}{https://github.com/Harrix/HarrixMathLibrary}

Данный оператор приводится в следующих источниках:

\begin{enumerate}
\item \cite{web:basegroup.ru:real_coded_ga} ---  \href{http://www.basegroup.ru/library/optimization/real_coded_ga/}{Непрерывные генетические алгоритмы - математический аппарат}.
\end{enumerate}