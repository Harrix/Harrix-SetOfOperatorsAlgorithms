\subsection{Равномерное скрещивание для вещественных векторов}\label{SetOfOperatorsAlgorithms:UniformCrossoverForReal}

Идентификатор: \textbf{UniformCrossoverForReal}.

Данный оператор скрещивания используется для вещественных векторов.

По сути ничем не отличается от \hyperref[SetOfOperatorsAlgorithms:UniformCrossover]{UniformCrossover}, кроме типа векторов, на котором работает оператор.

Пусть имеется два родителя (родительские хромосомы) $\overline{Parent}^1$ и $\overline{Parent}^2$. Потомок состоит из генов, каждый из которых выбран случайно из генов родителей на соответствующих позициях. То есть скрещивание происходит по формулам:

\begin{align}
\label{SetOfOperatorsAlgorithms:eq:UniformCrossoverForReal}
&Crossover \left( \overline{Parent}^1, \overline{Parent}^2, DataOfCros\right) = \overline{Offspring};\\
& \overline{Offspring}_i=Random\left( \left\lbrace \overline{Parent}^1_i;\overline{Parent}^2_i\right\rbrace \right), i=\overline{1,n} ;\nonumber\\
&\overline{Offspring}\in X.\nonumber
\end{align}

$ DataOfCros $ не содержит каких-либо параметров относительно данного типа скрещивания.

В \cite{web:basegroup.ru:real_coded_ga} данный вид скрещивания назван дискретным кроссовером (discrete crossover).

В библиотеке \textbf{HarrixMathLibrary} данная селекция реализуется через функцию \textbf{HML\_UniformCrossoverForReal}:

\href{https://github.com/Harrix/HarrixMathLibrary}{https://github.com/Harrix/HarrixMathLibrary}