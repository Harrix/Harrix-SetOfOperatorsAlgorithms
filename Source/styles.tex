%%% HarrixLaTeXDocumentTemplate
%%% Версия 1.18
%%% Шаблон документов в LaTeX на русском языке. Данный шаблон применяется в проектах HarrixTestFunctions, MathHarrixLibrary, Standard-Genetic-Algorithm  и др.
%%% https://github.com/Harrix/HarrixLaTeXDocumentTemplate
%%% Шаблон распространяется по лицензии Apache License, Version 2.0.

%%% Макет страницы %%%
\geometry{a4paper,top=2cm,bottom=2cm,left=2cm,right=1cm}

%%% Кодировки и шрифты %%%
%\renewcommand{\rmdefault}{ftm} % Включаем Times New Roman

%%% Выравнивание и переносы %%%
\sloppy
\clubpenalty=10000
\widowpenalty=10000

%%% Библиография %%%
\makeatletter
\bibliographystyle{utf8gost705u} % Оформляем библиографию в соответствии с ГОСТ 7.0.5
\renewcommand{\@biblabel}[1]{#1.} % Заменяем библиографию с квадратных скобок на точку:
\makeatother

%%% Изображения %%%
\graphicspath{{images/}} % Пути к изображениям
% Поменять двоеточние на точку в подписях к рисунку
\RequirePackage{caption}
\DeclareCaptionLabelSeparator{defffis}{. }
\captionsetup{justification=centering,labelsep=defffis}

%%% Цвета %%%
% Цвета для кода
\definecolor{string}{HTML}{B40000} % цвет строк в коде
\definecolor{comment}{HTML}{008000} % цвет комментариев в коде
\definecolor{keyword}{HTML}{1A00FF} % цвет ключевых слов в коде
\definecolor{morecomment}{HTML}{8000FF} % цвет include и других элементов в коде
\definecolor{сaptiontext}{HTML}{FFFFFF} % цвет текста заголовка в коде
\definecolor{сaptionbk}{HTML}{999999} % цвет фона заголовка в коде
\definecolor{bk}{HTML}{FFFFFF} % цвет фона в коде
\definecolor{frame}{HTML}{999999} % цвет рамки в коде
\definecolor{brackets}{HTML}{B40000} % цвет скобок в коде
% Цвета для гиперссылок
\definecolor{linkcolor}{HTML}{799B03} % цвет ссылок
\definecolor{urlcolor}{HTML}{799B03} % цвет гиперссылок
\definecolor{citecolor}{HTML}{799B03} % цвет гиперссылок
\definecolor{gray}{rgb}{0.4,0.4,0.4}
\definecolor{tableheadcolor}{HTML}{E5E5E5} % цвет шапки в таблицах
\definecolor{darkblue}{rgb}{0.0,0.0,0.6}
% Цвета для графиков
\definecolor{plotcoordinate}{HTML}{88969C}% цвет точек на координатых осях (минимум и максимум)
\definecolor{plotgrid}{HTML}{ECECEC} % цвет сетки
\definecolor{plotmain}{HTML}{97BBCD} % цвет основного графика
\definecolor{plotsecond}{HTML}{FF0000} % цвет второго графика, если графика только два
\definecolor{plotsecondgray}{HTML}{CCCCCC} % цвет второго графика, если графика только два. В сером виде.

%%% Отображение кода %%%
% Настройки отображения кода
\lstset{
language=C++, % Язык кода по умолчанию
morekeywords={*,...}, % если хотите добавить ключевые слова, то добавляйте
% Цвета
keywordstyle=\color{keyword}\ttfamily\bfseries,
%stringstyle=\color{string}\ttfamily,
stringstyle=\ttfamily\color{red!50!brown},
commentstyle=\color{comment}\ttfamily\itshape,
morecomment=[l][\color{morecomment}]{\#}, 
% Настройки отображения     
breaklines=true, % Перенос длинных строк
basicstyle=\ttfamily\footnotesize, % Шрифт для отображения кода
backgroundcolor=\color{bk}, % Цвет фона кода
frame=lrb,xleftmargin=\fboxsep,xrightmargin=-\fboxsep, % Рамка, подогнанная к заголовку
rulecolor=\color{frame}, % Цвет рамки
tabsize=3, % Размер табуляции в пробелах
% Настройка отображения номеров строк. Если не нужно, то удалите весь блок
%numbers=left, % Слева отображаются номера строк
%stepnumber=1, % Каждую строку нумеровать
%numbersep=5pt, % Отступ от кода 
%numberstyle=\small\color{black}, % Стиль написания номеров строк
% Для отображения русского языка
extendedchars=true,
literate={Ö}{{\"O}}1
  {Ä}{{\"A}}1
  {Ü}{{\"U}}1
  {ß}{{\ss}}1
  {ü}{{\"u}}1
  {ä}{{\"a}}1
  {ö}{{\"o}}1
  {~}{{\textasciitilde}}1
  {а}{{\selectfont\char224}}1
  {б}{{\selectfont\char225}}1
  {в}{{\selectfont\char226}}1
  {г}{{\selectfont\char227}}1
  {д}{{\selectfont\char228}}1
  {е}{{\selectfont\char229}}1
  {ё}{{\"e}}1
  {ж}{{\selectfont\char230}}1
  {з}{{\selectfont\char231}}1
  {и}{{\selectfont\char232}}1
  {й}{{\selectfont\char233}}1
  {к}{{\selectfont\char234}}1
  {л}{{\selectfont\char235}}1
  {м}{{\selectfont\char236}}1
  {н}{{\selectfont\char237}}1
  {о}{{\selectfont\char238}}1
  {п}{{\selectfont\char239}}1
  {р}{{\selectfont\char240}}1
  {с}{{\selectfont\char241}}1
  {т}{{\selectfont\char242}}1
  {у}{{\selectfont\char243}}1
  {ф}{{\selectfont\char244}}1
  {х}{{\selectfont\char245}}1
  {ц}{{\selectfont\char246}}1
  {ч}{{\selectfont\char247}}1
  {ш}{{\selectfont\char248}}1
  {щ}{{\selectfont\char249}}1
  {ъ}{{\selectfont\char250}}1
  {ы}{{\selectfont\char251}}1
  {ь}{{\selectfont\char252}}1
  {э}{{\selectfont\char253}}1
  {ю}{{\selectfont\char254}}1
  {я}{{\selectfont\char255}}1
  {А}{{\selectfont\char192}}1
  {Б}{{\selectfont\char193}}1
  {В}{{\selectfont\char194}}1
  {Г}{{\selectfont\char195}}1
  {Д}{{\selectfont\char196}}1
  {Е}{{\selectfont\char197}}1
  {Ё}{{\"E}}1
  {Ж}{{\selectfont\char198}}1
  {З}{{\selectfont\char199}}1
  {И}{{\selectfont\char200}}1
  {Й}{{\selectfont\char201}}1
  {К}{{\selectfont\char202}}1
  {Л}{{\selectfont\char203}}1
  {М}{{\selectfont\char204}}1
  {Н}{{\selectfont\char205}}1
  {О}{{\selectfont\char206}}1
  {П}{{\selectfont\char207}}1
  {Р}{{\selectfont\char208}}1
  {С}{{\selectfont\char209}}1
  {Т}{{\selectfont\char210}}1
  {У}{{\selectfont\char211}}1
  {Ф}{{\selectfont\char212}}1
  {Х}{{\selectfont\char213}}1
  {Ц}{{\selectfont\char214}}1
  {Ч}{{\selectfont\char215}}1
  {Ш}{{\selectfont\char216}}1
  {Щ}{{\selectfont\char217}}1
  {Ъ}{{\selectfont\char218}}1
  {Ы}{{\selectfont\char219}}1
  {Ь}{{\selectfont\char220}}1
  {Э}{{\selectfont\char221}}1
  {Ю}{{\selectfont\char222}}1
  {Я}{{\selectfont\char223}}1
  {і}{{\selectfont\char105}}1
  {ї}{{\selectfont\char168}}1
  {є}{{\selectfont\char185}}1
  {ґ}{{\selectfont\char160}}1
  {І}{{\selectfont\char73}}1
  {Ї}{{\selectfont\char136}}1
  {Є}{{\selectfont\char153}}1
  {Ґ}{{\selectfont\char128}}1
  {\{}{{{\color{brackets}\{}}}1 % Цвет скобок {
  {\}}{{{\color{brackets}\}}}}1 % Цвет скобок }
}
% Для настройки заголовка кода
\DeclareCaptionFont{white}{\color{сaptiontext}}
\DeclareCaptionFormat{listing}{\parbox{\linewidth}{\colorbox{сaptionbk}{\parbox{\linewidth}{#1#2#3}}\vskip-4pt}}
\captionsetup[lstlisting]{format=listing,labelfont=white,textfont=white}
\renewcommand{\lstlistingname}{Код} % Переименование Listings в нужное именование структуры
% Для отображения кода формата xml
\lstdefinelanguage{XML}
{
  morestring=[s]{"}{"},
  morecomment=[s]{?}{?},
  morecomment=[s]{!--}{--},
  commentstyle=\color{comment},
  moredelim=[s][\color{black}]{>}{<},
  moredelim=[s][\color{red}]{\ }{=},
  stringstyle=\color{string},
  identifierstyle=\color{keyword}
}

%%% Гиперссылки %%%
\hypersetup{pdfstartview=FitH,  linkcolor=linkcolor,urlcolor=urlcolor,citecolor=citecolor, colorlinks=true}

%%%  Оформление абзацев %%%
\setlength{\parskip}{0.3cm} % отступы между абзацами
% оформление списков
\setlist{leftmargin=1.5cm,topsep=0pt}

%%% Псевдокоды %%%
% Добавляем свои блоки
\makeatletter
\algblock[ALGORITHMBLOCK]{BeginAlgorithm}{EndAlgorithm}
\algblock[BLOCK]{BeginBlock}{EndBlock}
\makeatother

% Нумерация блоков
\usepackage{caption}% http://ctan.org/pkg/caption
\captionsetup[ruled]{labelsep=period}
\makeatletter
\@addtoreset{algorithm}{chapter}% algorithm counter resets every chapter
\makeatother
\renewcommand{\thealgorithm}{\thechapter.\arabic{algorithm}}% Algorithm # is <chapter>.<algorithm>

%%% Формулы %%%
%Дублирование символа при переносе
\newcommand{\hmm}[1]{#1\nobreak\discretionary{}{\hbox{\ensuremath{#1}}}{}}

%%% Таблицы %%%
% Раздвигаем в таблице без границ отступы между строками в новой команде
\newenvironment{tabularwide}%
{\setlength{\extrarowheight}{0.3cm}\begin{tabular}{  p{\dimexpr 0.45\linewidth-2\tabcolsep} p{\dimexpr 0.55\linewidth-2\tabcolsep}  }}  {\end{tabular}}
\newenvironment{tabularwide08}%
{\setlength{\extrarowheight}{0.3cm}\begin{tabular}{  p{\dimexpr 0.8\linewidth-2\tabcolsep} p{\dimexpr 0.2\linewidth-2\tabcolsep}  }}  {\end{tabular}}

% Многострочная ячейка в таблице
\newcommand{\specialcell}[2][c]{%
  {\renewcommand{\arraystretch}{1}\begin{tabular}[t]{@{}l@{}}#2\end{tabular}}}

% Многострочная ячейка, где текст не может выйти за границы
\newcolumntype{P}[1]{>{\raggedright\arraybackslash}p{#1}}
\newcommand{\specialcelltwoin}[2][c]{%
  {\renewcommand{\arraystretch}{1}\begin{tabular}[t]{@{}P{1.98in}@{}}#2\end{tabular}}}
  
% Команда для переворачивания текста в ячейке таблицы на 90 градусов
\newcommand*\rot{\rotatebox{90}}
  
%%% Абзацы %%%
% Отсупы между строками
\singlespacing

%%% Рисование графиков %%%
\pgfplotsset{
every axis legend/.append style={at={(0.5,-0.13)},anchor=north,legend cell align=left},
tick label style={font=\tiny\scriptsize},
label style={font=\scriptsize},
legend style={font=\scriptsize},
grid=both,
minor tick num=2,
major grid style={plotgrid},
minor grid style={plotgrid},
axis lines=left,
legend style={draw=none},
/pgf/number format/.cd,
1000 sep={}
}
% Карта цвета для трехмерных графиков в стиле графиков Mathcad
\pgfplotsset{
/pgfplots/colormap={mathcad}{rgb255(0cm)=(76,0,128) rgb255(2cm)=(0,14,147) rgb255(4cm)=(0,173,171) rgb255(6cm)=(32,205,0) rgb255(8cm)=(229,222,0) rgb255(10cm)=(255,13,0)}
}
% Карта цвета для трехмерных графиков в стиле графиков Matlab
\pgfplotsset{
/pgfplots/colormap={matlab}{rgb255(0cm)=(0,0,128) rgb255(1cm)=(0,0,255) rgb255(3cm)=(0,255,255) rgb255(5cm)=(255,255,0) rgb255(7cm)=(255,0,0) rgb255(8cm)=(128,0,0)}
}

%%% Разное %%%
% Галочка для отмечания в тескте вариантов как OK
\def\checkmark{\tikz\fill[scale=0.3](0,.35) -- (.25,0) -- (1,.7) -- (.25,.15) -- cycle;} 